\documentclass[letterpaper,10pt]{article}

\usepackage[utf8]{inputenc}
\usepackage{tocloft}                        %Modify table of contents (toc)      
\renewcommand{\contentsname}{\hfill\bfseries\Large Contents\hfill}   %center toc title
\renewcommand{\cftaftertoctitle}{\hfill}
\renewcommand{\cftsecleader}{\cftdotfill{\cftdotsep}}   %put .... in toc
\usepackage{url}                            %put websites in paper
\usepackage[margin=0.8in]{geometry}         %adjust margins
\usepackage{graphicx}                       %use images
\graphicspath{ {./Figures/} } %where to find files
\usepackage[outdir=./Figures/]{epstopdf}
\usepackage{sectsty}                        %center section titles
\allsectionsfont{\centering}
\usepackage{chngcntr}                       %Start Figure counting according to section
\counterwithin{figure}{section}             %where to start that counter
\usepackage[section]{placeins}              %able to use Floatbarrier
\usepackage{caption}
\usepackage{subcaption}                     %subfigures allowed
\usepackage{amsmath}                        %math stuff
\usepackage{verbatim}                        %block commenting

\renewcommand{\thefootnote}{\fnsymbol{footnote}} % modify footnotes to symbols
%opening
\title{Calibration of the Preshower Calorimeter}
\author{N. Compton, C. Smith, and K. Hicks}


\begin{document}

\maketitle

\begin{abstract}
The CLAS12 Pre-Shower Calorimeter (PCAL) has been described geometrically. 
This information is used to correct for the light attenuation curve seen by the scintillators. 
The ADC readout plotted as a function of distance away from the PMT, gives the form of the light attenuation. 
This form is fit and the parameters are recorded to the CLAS12 database. 
The process used to obtain these parameters are described in this note. 
\end{abstract}

\tableofcontents
\clearpage

\input{Design.tex}

\input{VariousCuts.tex}

\input{FitADC.tex}

\input{LightAtt.tex}

\input{Studies.tex}

\input{Bibliography.tex}



\end{document}
